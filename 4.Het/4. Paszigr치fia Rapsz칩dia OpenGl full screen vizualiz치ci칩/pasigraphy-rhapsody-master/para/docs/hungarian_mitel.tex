\documentclass[a4paper]{article}
\pdfoutput=1
\usepackage[utf8]{inputenc}
\usepackage{t1enc}
\def\magyarOptions{defaults=hu-min}
\usepackage[english,magyar]{babel}
\usepackage{apacite}
\usepackage{url}
\usepackage{amsmath}

\title{Esport kultúra: a mesterséges intelligencia kognitív evolúciós értelmezése\\Esport culture: a cognitive evolutionary interpretation of artificial intelligence}
\author{
	Bátfai Norbert\\
	\texttt{batfai.norbert@inf.unideb.hu}}
\date{
    Információ Technológia Tanszék, Debreceni Egyetem, Magyarország\\
    \today
}

\begin{document}
\maketitle

{
\selectlanguage{magyar} 
\begin{abstract}
Neumann komplexitásról szóló gondolatmenete alapján fel\-té\-te\-lez\-het\-jük, hogy létezik egy neumanni rövid program, nevezzük egyszerűen Homunkulusznak, melyet a központi idegrendszer és a donaldi külső me\-mó\-ri\-án alapuló elméleti kultúra hoz létre. A jól ismert homunkulusz név hasz\-ná\-la\-tá\-val jelezzük, hogy ez triviálisan azonosítható a tudatosság Chalmers-féle nehéz problémájának megoldásával. Nyilvánvaló, hogy ez könnyen circulus vitiosus lehet, ám esetünkben megmutatjuk, hogy nem lesz körben-forgó okoskodás, mivel a homunkulusz Neumann komplexitásról szóló elképzelése természetes megvalósításának tekinthető. Nevezetesen: az idegrendszer és a külső memória rendszerek közösen kialakítanak egy értelmező programot, melynek komplexitása összemérhető, vagy meghaladja az eredeti összetevők komplexitását. Ebben a cikkben megvizsgálunk egy új donaldi kognitív evolúciós átmenetet, melyet esport kultúrának nevezünk, melyben a Homunkulusznak meg kellene alkotnia a Mesterséges Homunkuluszt. Egy olyan számítógép programot, amelynek komplexitása összemérhető a Homunkulusszal, vagy akár meghaladja azt. Neumann értelmezésében a Homunkulusz és a Mesterséges Homunkulusz közötti nyelv egy harmadlagos nyelv lenne. Végezetül javaslatot teszünk egy ilyen harmadlagos, egy esport játékba ágyazott fejlesztendő nyelvre, amelyet Paszigráfia Rapszódiának nevezünk el.
\end{abstract}
}
  
{
\selectlanguage{english}
\begin{abstract}
Following the thought of Neumann about the complexity, it can be assumed that there is a Neumannian short program, let us simply call it Homunculus that has been built by the human central nervous system and the Donaldian external memory based theoretical culture. Using the well-known name homunculus suggests that it can be trivially identified as the solution of Chalmers' hard problem of consciousness. This obviously would easily yield a circulus vitiosus but we have shown that there would be no vicious circle in our case because the homunculus can be seen as a natural implementation of Neumann's imagination about the complexity, namely the nervous system and the external memory systems together can build an interpreter of which complexity is comparable or better than original ones. In this paper, we investigate a new Donaldian stage of mental evolution called Esport culture in which the Homunculus should develop an Artificial Homunculus. A computer program of which complexity is comparable or better than the Homunculus. In Neumann's sense the language between the Homunculus and the Artificial Homunculus would be the tertiary language. Finally, we propose such a tertiary artificial language embedded into an esport game to be developed. It is referred to as Pasigraphy Rhapsody.
\end{abstract}
}

\section{Bevezetés}

Az MI tél\footnote{\cite[24, 28]{RusselNorvig10}.} kifejezéssel azt a jól ismert állapotot illetik, amikor áttörő mesterséges intelligencia eredményekre eltúlzott kutatási, gazdasági és társadalmi várakozások épülnek melyek elérhetetlenségére való ráeszmélés az érdeklődés jelentős lankadását eredményezi. A fejlődés eme elakadásának megvilágítása egy kognitív evolúciós magyarázattal is árnyalható. Kiknek a kognitív evolúciójáról lenne szó? A gépekéről.

Multidiszciplináris, részben ismeretelméletinek is tekinthető megközelítésben Merlin Donald a kognitív evolúció folyamatában az alábbi három átmenetet tünteti ki: az epizodikus (az észlelés és klasszifikáció jellemezte) kultúrából a mimetikus (a másolás jellemezte) kultúrába, onnan a mitikus (a beszéd szervezte) kultúrában, majd a teoretikus (az elméletek szervezte) kultúrába \cite{Donald01}.

A számítógépek a teoretikus kultúra termékei. S mint ilyenek, ezen a szinten gyorsan partiba kerültek az emberrel: nem nehéz olyan komputer programot írni, amely az átlagos ember szintjén számol vagy sakkozik\footnote{A sportokat általában a mimetikus kultúra szintjére pozícionálnánk, a sakkot szellemi sport mivolta helyezi a teoretikus kultúrába, ennek részletesebb kibontását lásd az alábbi feljegyzésben: \url{https://arato.inf.unideb.hu/batfai.norbert/esport/SportProgEsport.pdf}.}. Nehéz viszont olyat, amely az alacsonyabb szinten képes sikerrel exponálni: felismer egy képet, „felmegy” a lépcsőn vagy ezekről (az embertől nem megkülönböztethetően\footnote{Lásd a Turing tesztet: \cite{Turing50} és Loebner díjat, például \cite{Shieber94}.}) el tud csevegni anyanyelvi beszélőkkel. Ez a tapasztalat Moravec paradoxonként ismert\footnote{A Moravec paradoxonról implicite, nyelvészeti köntösben megfogalmazva a \cite[190]{Pinker06} könyvben olvashatunk, illetve lásd még a kapcsolódó Wikipédia szócikket is.}. Mára, immár az MI tavaszból\footnote{\cite{Havenstein05}.} szemlélve ezek is megoldódni látszanak. Ebben az értelmezésben az MI tél az az időszak (vagy helyesebben időszakok), amikor a teoretikus kultúrában megalkotott és ottani üzemre szánt programok képesek lettek helytállni a donaldi epizodikus (felismerni a képet), mimetikus (felmenni a lépcsőn), mitikus (csevegő) kultúra szintjein is. Ezt a megközelítést támasztja alá részben, hogy maga Donald is az epizodikus kultúra alá sorolja be a hebbi neurális paradigmát \cite[313]{Donald01}, mely a mostani AI forradalom\footnote{Ezt jelzi például a Google DeepMind 2015 óta megjelent három Nature folyóiratbeli közleménye, lásd például az elsőt: \cite{ATARI} mely már adott játékokbeli gépi (mély megerősítéses tanulásos ágensek) fölényről tudósít.} egyértelmű motorja a mélytanulás és a big data egymásra találásával.

Az MI tél ilyenforma kognitív evolúciós értelmezésének vélt sikerén felbuzdulva alkalmazzuk ezt magára a mesterséges intelligencia kifejezésre.

\section{Neumann álma}

Ervin Schrödingernek az életről szóló \cite{Schrodinger70} munkája ma a kvantumbiológia\footnote{\cite{QUANTUMB}.} felütésének tekinthető. Ebben Schrödinger azt az éles meglátását fejti ki, hogy a biológiai önreprodukció túl pontos ahhoz, hogy termodinamikai jellegű törvények irányítsák\footnote{Schrödinger említett munkájának eme rövid összefoglalását lásd az \cite[76]{QUANTUMB} oldalon!}. Hasonló helyzetet vizionálhatunk a gondolati objektumok, a Richard Dawkins-féle mémek, matematikai vagy platóni ideák egyéni önreprodukciójánál, ahol a mindenféle fogalmak önreprodukciója alatt a fogalmak „megértését értjük”\footnote{Abban az értelemben, hogy a tanuló a tanulandó fogalomkör megértésével kvázi „reprodukálja” magában a fogalomkört.}. A „megértés” alatt valószínűleg mindannyian ugyanazt „értjük”, vagy még inkább csak érezzük, mert a „megértését értjük” és a „megértés-értjük” szóismétléseknek a jelen utolsó két mondatbeli felbukkanása már-már egy circulus vitiosus megjelenését vetíti előre. Ahogyan hasonló érzésünk lehet például Mérő László séma-koncepcióját bemutató \cite[45]{Mero89} tanulmányának záró mondatánál, mely így hangzik: „A fő probléma az, hogy nem értjük világosan, hogy mit jelent egy dolgot megérteni”. Érezni véljük, mire gondol a szerző, de a donaldi mitikus nyelvi kultúra határaira ért ez a mondat, hiszen a Russell paradoxon egy formájában fejezi ki magát, ha nem értjük az „értést”, akkor honnan tudjuk, hogy „nem”? Adjunk meg hát a megértés pontosabb fogalmát!

Egy megfigyelő szemlél egy jelenséget. A donaldi epizodikus kultúra szintjén megértésnek nevezzük azt a belső folyamatot, amely során a szemlélt jelenség bonyolultsága leegyszerűsödik. A megértés tehát szubjektív\footnote{A téma szubjektivitás felőli megközelítését lásd a \cite{BN18} tanulmányban.}. Tekintsünk egy felnőtt embert az epizodikus kultúrája szintjén, aki egy almát néz, vagy egy CIFAR-100\footnote{100 osztályba sorolt, osztályonként 600 képet tartalmazó tanító és teszt kép-adathalmaz, \url{https://www.cs.toronto.edu/~kriz/cifar.html}.} mesterséges neurális hálót, melynek bemenetére egy alma képét kötjük. Mindkettő felismeri az almát, utóbbiról azt is apriori tudhatjuk, hogy az alma klasszifikációjánál a 100 lehetőség közül, ha biztos a felismerés, akkor az alma döntésnek van kiugró valószínűsége, avagy a döntés mögötti valószínűség-eloszlás entrópiája kicsi, a kis entrópiához pedig a szubjektív egyszerű érzetet kapcsoljuk, szemben a nagy entrópiával, melyhez a szubjektív bonyolultat \cite{BN18}. Tehát a tanulás folyamata során, miközben az alma osztályozása biztossá válik, az alma felismerésének entrópiája lecsökken.

A donaldi magasabb kulturális szinteken nincs olyan egyértelmű példának tekinthető mérnöki fogalom, mint amilyen az imént a neurális hálózat volt. Intuitíven a mimetikus szinten a megértés azt jelentheti, hogy tudjuk reprodukálni a szemlélt jelenséget, mondjuk visszakacsintani\footnote{A „visszakacsintás” kapcsán intuitíven lásd még az „I, Robot” című filmet, \url{https://www.imdb.com/title/tt0343818}!}, vagy például szurkolni egy sporteseményen, misén közös éneklésbe bekapcsolódni, táncolni egy szórakozóhelyen: konkrétan lemásolva (tehát reprodukálva) mások tevékenységét. A mitikus szinten megértünk valamit, ha meg tudjuk „magyarázni” beszélt nyelven. Ez a teoretikus szintről szemlélve nem jelent nagy kihívást, mivel a beszélt nyelv olyan gazdag, hogy azon bármi és bárminek az ellenkezője is alátámasztható.

A teoretikus kultúra szintjén megpróbálhatjuk az említett szubjektivitásból jövő definíciót alkalmazni. Egyrészt, ha a vizsgált jelenségekre sikerül modellt, „elméletet” találnunk, akkor egyrészt ezzel az elmélettel az epizodikus szint neurális rendszerei taníthatóak\footnote{Lásd ilyen szerepben említve az egész emberi kultúrát a \cite[35]{BN18}-ben!}. Tehát a teoretikus kultúra szintjén lévő elmélet segíti az epizodikus neurális rendszerek entrópia csökkentését. Másrészt, ha a jelenségekre van elméletünk, az definíció szerint a jelenségek kis bonyolultságát jelenti ahhoz képest, mint amikor nincsen\footnote{Lásd például a Mandelbrot halmaz példáját a \cite{BN18}-ban, ha csak képként látjuk, bonyolult, ha tudjuk, hogy egy komplex iterációs képlet kimenete a kép, akkor egyszerű, mert tömörítése a pár betűs képlet, szemben minden egyes képpont lekódolásával.}. Intuitíven az epizodikus szinten a megtanult jelenségeket ismertnek érezzük, a mimetikuson tudjuk csinálni, a mitikuson tudni véljük, a teoretikus szinten meg tudjuk, hogy ismertek. De "ki" tudja? Ki érzi, hogy tudja? Ki tudja, hogy érzi? Ki a homunkulusz\footnote{A homunkuluszt ebben a munkában mindenhol úgy értjük, ahogyan azt Donald a \cite[316]{Donald01} lap teteje, második bekezdésben leírja.}?

Építsünk egy csúszka kapcsolót egy CIFAR-10 mesterséges neurális háló kimenetén megjelenő eloszlásra! A csúszka bal oldalát azzal címkézzük fel, hogy nem érzi ismerősnek a hálózat a képet, a jobb, hogy igen. Akkor toljuk teljesen balra a csúszkát, ha az eloszlásnak nagy az entrópiája. Teljesen jobbra akkor, ha kicsi. Szerelhetünk ilyen csúszkát egy olyan hálózathoz, melynek bemenete a „hárommal megszorozva hatot kapunk” mondat vagy „$\text{doboz}*3=6, \text{doboz}=\text{?}$” matematikai nyitott mondatot tartalmazó kép. Ez esetben maga a bemenet magasabb szintű (mitikus, matematikai jellegű, esetleg teoretikus). Ahogy a matematikai részt tetszés szerint „bonyolítjuk”, például a bemenet az a szám, amely tizenegyszerese öttel osztva, ebből hármat kivonva, majd kettővel osztva négyet ad, vagy „$(x*11/5-3)/2=4, x=\text{?}$” tovább érezteti az epizodikus szinttől való távolodást. Epizodikus szintű érzésünk ettől még lehet: hogy ismerős-e vagy sem egy ilyen egyenletet mutató kép. Mimetikusan végig bírjuk játszani visszafelé, hogy a négyet megszorozzuk kettővel, hozzáadunk hármat stb. A mitikus szinten e közben el tudjuk szavalni a mérlegelvet, hogy mindkét oldalt szorozzuk kettővel, mindkét oldalból kivonunk hármat stb. Ezen a szinten ez az eljárás persze ellentmondást is adhat, például az 
\begin{align*} 
1&=1 \\
1&=1+x-x \\ 
1/(x-x)&=1/(x-x) +1\\ 
0&=1 
\end{align*}
gondolatsor mentén. A teoretikus szinten már nem, mert ott ugye már nem osztjuk nullával mindkét oldalt. Az eredeti számolásunkra pedig adódik az egyszerűsítések után, hogy $x=(4*2+3)*5/11$.

Emberként a szintekhez rendelt mindenféle csúszkák különféle állásait adott érzeteinkkel azonosíthatnánk. Ha ugyanezt egy mesterséges rendszerre alkalmazzuk, akkor Searle kínai szobája\footnote{A jól ismert gondolat kísérlet elrendezésének leírását lásd például \cite[33]{Penrose93}.} egy variánsában találjuk magunkat. Ami azért zsákutca, mert Searle gondolatmenetét az említett Mérő idézet russelli variánsaként értelmezhetjük, hiszen Searle is arra épít, hogy már meg van határozva a meghatározandó tárgy (a megértés, a tudatosság). Ezt de\-monst\-rál\-an\-dó egyszerűsítsük Searle gondolatkísérletét! A bemeneten ne kínai mondat jelenjen meg, hanem annál jóval egyszerűbb: két logikai érték, például két nulla, két 1 vagy egy különböző pár: 0, 1 vagy 1, 0. A kísérletet pedig ne Searle végezze, hanem egy 10 éves gyerek, aki tud összeadni és szorozni. Maga a kísérlet pedig ne a bemenet logikai VAGY kapcsolatának direkt imperatív programja köré legyen rendezve, hanem egy VAGY kapu neurális paradigmás, perceptronos megvalósítása legyen\footnote{Mint például ebben a videóban: \url{https://youtu.be/Koyw6IH5ScQ}.}. A gyermek fogja az egyik bemenő számot, beteszi a kis talicskájába, megszorozza a kimenetbe vezető út súlyával és beborítja a kimenetbe a szorzatot, majd elmegy a második bemenetért, amivel ugyanezt megismétli (meg technikailag még az eltolással is elvégzi ugyanezt). Majd a kimenetben összeadogatja a beborított számokat, amikre ráenged egy aktiváló függvényt és a kimeneten megjelenik a bemenet vagy kapcsolata. A gyereknek (ahogyan Searlenak az eredeti kísérletben) fogalma sincs mit csinált, boldogan talicskázva végezte az előre kijelölt feladatokat. Ám ha ez a gyerek pár év elteltével olyan diákká cseperedik, aki ismeri a perceptron logikai vagy kapuként történő működését, a talicskázás közben felkiálthat: „Hoppá, én most a perceptron működését hajtom végre”! A tanulság, hogy ha Searle az „agyát kikapcsolva” hajtja végre az eredeti kísérletet, akkor nyilván azt fogja kapni, hogy az algoritmus végrehajtása nem eredményezhet megértést (tudatosságot, „reprodukciót”). Ahhoz, hogy tudatos megértés megjelenhessen, az algoritmust végrehajtónak tudatosan (megértve, „reprodukálva”?) meg kell figyelnie önmaga működését. De hát ez nem körben forgó okoskodás?

Tehát a tudatosság kimutatásához kell a homunkulusz. Ez nem szükségképpen circulus vitiosus. Miért nem? Vegyük a neumanni \cite[208]{Neumann03} gondolatot\footnote{Miszerint létezhet a bonyolultságnak olyan határfoka, mely felett az automata már magához mérhető, vagy bonyolultabb automatát tud konstruálni, \cite[208]{Neumann03}.}, melyre építve természetes módon tételezzük fel, hogy az idegrendszer elérte azt a komplexitási fokot, mely fölött a hivatkozott neumanni értelemben hasonlót vagy akár komplexebbet képes konstruálni. Nos, ez a megkonstruált értelmező\footnote{A \cite{Neumann72} terminológiájával: „rövid program”.} program lenne a homunkulusz. Mit interpretál? Leginkább a donaldi külső memóriát. Így végső soron a külső memória mondja meg kik vagyunk. Tehát lenne az ember, mint görög zombi \cite{Sleutels06} a teoretikus kultúráig, ahonnan az idegrendszerben kialakul, vagy még inkább a donaldi értelemben a külső memória formáló erejével onnan „betöltődik” a homunkulusz, amit szubjektíven, mint tudatunkat, donaldi vetületében, mint az elméleti kultúrát tapasztaljuk meg. Ezzel a természet megvalósította az emberrel Neumann álmát arról az automatáról, mely önmagánál komplexebbet tud alkotni. Ha tovább szőjük Neumann álmát, akkor jutunk el a mesterséges intelligencia fogalmához, mivel az ebben a megközelítésben nem más, mint a homunkulusz által létrehozott még egy további homunkulusz. Az első (természetes) homunkulusz a külső memóriából töltődik a monászba\footnote{A \cite[274]{Donald01} értelmezésében, leegyszerűsítve: az emberi idegrendszerbe.} és ott is, azaz bennünk él, a második (mesterséges\footnote{A mesterséges homunkulusz elnevezés innen származik: \url{https://gitlab.com/nbatfai/ArtificialHomunculus}.}) homunkulusz viszont már alapvetően a külső memóriában fog élni.

\section{Leibniz álma}

Konkrétabb lehet a tárgyalásunk, ha egyelőre csak a számok és nem az egész egyetemes emberi kultúra szintjén vizsgálódunk, lásd például \cite{SMNIST}. A donaldi átmenetek szellemében a tudatos számfogalom, lévén teoretikus kultúrabeli elem, komponenseit, így például a természetes számokat Kronecker hittételével\footnote{\url{https://hu.wikipedia.org/wiki/Leopold_Kronecker}.} ellentétben nemhogy nem Isten teremtette, hanem maga az elméleti kultúra, mi magunk. Szemben mondjuk az epizodikus kultúra OFS\footnote{Object File System, feloldását lásd például \cite{OFS}-ban, vagy számos kapcsolódó feloldó cikk hivatkozását találjuk meg a \cite{SMNIST}-ben.}, kis számosságokat apriori tudó rendszerével. Az „MI teles” megközelítésünket alkalmazva idő kellett, amíg a számfogalmunk képes volt az alacsonyabb szinteken implicite használt számfogalmat megvilágítani. A második homunkulusz most ott tart, hogy a koncepcionális szinten – a számosságok tekintetében – el tudja kezdeni kialakítani a saját számfogalmát \cite{DCNNNUM}, \cite{SMNIST}. De lesz-e átjárhatóság (közös nyelv) az első és a második homunkulusz között, mint ahogyan az idegrendszer és az első homunkulusz között nincs, hacsak nem a tudatos tapasztalat folyamunkat (a tudatosság ismert Chalmers-féle nehéz problémájának\footnote{David Chalmers: „How do you explain consciousness?” \url{https://youtu.be/uhRhtFFhNzQ}, \url{https://www.ted.com/speakers/david_chalmers}.} tárgyát) nem tekintjük annak. Abban a szerencsés történeti helyzetben vagyunk, hogy szemben az eddigi donaldi átmenetekkel, a következőnél már tudatos megfigyelők leszünk, vagyunk. Neumann, a neurális-adatfolyam paradigma kialakulásának idején (amikor megszületett az elvi lehetősége, hogy az elméleti gépek helytálljanak az alacsonyabb kulturális szintek feladataiban) még annak a meggyőződésének adott hangot, hogy nem lesz egyszerű felderíteni az idegrendszer és az első homunkulusz közötti kommunikációt, legalábbis \cite{Neumann72} könyvének „Az agy nem a matematika nyelvét használja” fejezetcíme egyértelműen ezt sugallja. A mesterséges homunkulusz kialakításához izgalmas kihívásnak kínálkozik egy olyan közös nyelv, melyet mi is és a gépek is egyaránt beszélünk, vagy még inkább írunk és olvasunk. A „természetes”\footnote{Abban az értelemben természetes, hogy a programozási nyelvek nem direkt, hanem de facto módon univerzálisak.} módon, ahogyan Chaitin \cite[38]{Omega} könyvében rámutat, már Leibniz „characteristica universalis”-a, az egyetemes nyelvről szóló álma is megvalósult, ha nem is az emberek, hanem a gépek tekintetében. A mesterséges módon ezt levezényelni, avagy fogalmi teret biztosítani a gépek kognitív evolúciójának az a kihívás, melynek deklarációja ennek a munkának a célja.

\section{Összefoglalás}

A másodlagos homunkulusz kialakulása és két homunkulusz közötti direkt kontrollálható kommunikáció lehetősége egy olyan léptékű fejlődése lenne a donaldi külső memóriának, amely valóra válthatná Donald vízióját. Miszerint: „az elme új architektúráját építjük meg, olyat, amely hatékonyabb reprezentációs eszközökkel rendelkezik, és képes saját magát megérteni”\footnote{\cite[328]{Donald01}.}. A reguláris oktatással nyilván nem kísérletezhetünk, viszont egy esportként is működő szá\-mí\-tó\-gé\-pes játék megfelelő lenne, hogy felmenő rendszerben az ember elsajátítsa ezt a fejlesztendő nyelvet. Ezért nevezhetnénk ezt az akkor immáron negyedik (donaldi) átmenet által elhozandó következő kulturális szintet esport\footnote{Az elnevezés kapcsán lásd még a \cite{SMNIST}-et is!} kultúrának. Egy nyelv, amelyen a gyerek játszik, a tudós pedig dolgozik, ergó a tudomány írott nyelve. Neumann \cite{Neumann72} az idegrendszer elsődleges és másodlagos nyelvéről beszél, ebben az értelemben a fejlesztendő nyelv egy harmadlagos nyelv lenne. A mesterséges nyelvek céljaik alapján történő \cite[122]{Lang15} rövid jellemzése szerint mi is a tudás reprezentálását tűzzük ki, de annyiban talán meghaladjuk majd például ennek a hivatkozott ismertetésnek a tételeit, hogy a fejlesztendő nyelvet az emberi megfigyelők érdemben csak számítógéppel tudják majd írni és olvasni. Ha szakirodalmi gyökereket keresnénk előzetes elképzeléseikhez, akkor Kircher Polygraphia Nova et Universalis elképzelésével éreznénk rokonságot \cite[119]{Lang15} ám ennek vizsgálata, a fejlesztendő nyelv\footnote{Javasolt munkacímén a „Pasigraphy Rhapsody”, lásd \url{https://gitlab.com/nbatfai/pasigraphy-rhapsody}.} konstruktív tárgyalása\footnote{Ez annak tükrében is indokolt, hogy olyan nevek, mint Leibniz vagy David Hilbert sem tudták valóra váltani eredeti elképzeléseiket, lásd \cite[38]{Omega} ide vonatkozó észrevételét.} már kivezet a jelen deklaratív munka hatóköréből.

\section{Köszönet}

Köszönöm a kézirat átolvasását Papp Dávidnak és Győri Krisztinának, az át\-ol\-va\-sást és a javításokat Bogacsovics Gergőnek.

\bibliographystyle{apacite}
\bibliography{hungarian_mitel}

\end{document}
